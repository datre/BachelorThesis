\chapter{kapitel 2}
\section{IDE einrichten}
http://stackoverflow.com/questions/33676829/vim-configuration-for-linux-kernel-development
http://vim.wikia.com/wiki/Vim\_buffer\_FAQ
\section{Function tracing}
Um herauszufinden, welche Funktionen beim 'Tegra TK1' für die UART-Baudrate zuständig ist, habe ich mir die relevanten Funktionen ausgeben lassen. Da als Plattform 'ARM' eingesetzt wird, befindet sich im Ordner 'arch/arm/boot/dts' im Linux Kernel Projekt die Namen, welche anzeigen, für welches Bauteil welcher Treiber bei welcher Adresse benutzt wird. In der Datei 'tegra124.dtsi' lässt sich dann herausfinden, dass der Uart-Treiber auf dem '8250'-Chip basiert. In 'drivers/tty/serial/8250' sind die relevanten Quelldateien vorhanden. Um zu sehen, wie der Linux-Kernel die Baudrate ändert, lässt sich als zu filternder Funktionsnamenspräfix \enquote{serial8250} ausmachen. 	

Dafür müssen im Linux Kernel unter \enquote{Kernel Hacking -> Tracers} die Optionen für \enquote{Kernel Function Tracer}, \enquote{Kernel Function Graph Tracer} und \enquote{enable/disable function tracing dynamically} auf \enquote{true} gesetzt werden. Anschließend ist es möglich, die Traces in einen eingehängten Ordner schreiben zu lassen. Hierfür kann man entweder das Dateisystem zur Laufzeit mittels 'mount -t debugfs nodev /sys/kernel/debug' oder zum Betriebssystemstart mittels \enquote{echo "debugfs       /sys/kernel/debug          debugfs defaults        0       0" >> /etc/fstab} einhängen. 

Danach schalte ich das Tracing mittels \enquote{echo 0 > /sys/kernel/debug/tracing/tracing\_on} aus, um irrelevante Einträge zu vermeiden. 

echo function\_graph > current\_tracer

echo serial8250* > set\_ftrace\_filter

echo 1 > tracing\_on

cat trace\_pipe

\section{Uart driver}
Speziellen divisor setzen:
Clk speed / baud << 4
Sense Value register is updated with the count value. The low 20
bits of the ASR give the number of clocks within a single bit. Because the UART uses 16x oversampling, the resulting value
needs to be adjusted by shifting right 4 bits, then loading the resulting count in the divisor latch of the UART. (In the code
snippet below, the lower 4 bits are rounded to give slightly greater accuracy.)
(Technical Reference Manual 2408)
408000000 / 16 * 115200 (1843200) = 221

\section{rtems}
Kein Support für Tegra
git clone git://git.rtems.org/rtems.git rtems 
cd rtems
\subsection{rtems-source-builder}
mkdir src
cd src
git clone git://git.rtems.org/rtems-source-builder.git
cd rtems-source-builder
source-builder/sb-check
../source-builder/sb-set-builder --log=l-arm.txt --prefix=\$(pwd)/4.12 4.12/rtems-arm

\section{Linaro-Hikey}
Zwei Anleitungen führen zum Ziel, eine zur Verzweiflung XD
Erst uefi, dann der Rest:
https://github.com/96boards/documentation/wiki/HiKeyUEFI
http://wiki.lemaker.org/HiKey(LeMaker\_version):Building\_Debian\_from\_Source\_Code
dir=\$(pwd); cd /usr/bin; sudo mv python python.bck; sudo ln -s python2.7 python; cd \$dir; \${UEFI\_TOOLS\_DIR}/uefi-build.sh -b RELEASE -a ../arm-trusted-firmware hikey -c ../LinaroPkg/platforms.config; cd /usr/bin/; sudo mv python.bck python; cd \$dir

\section{Notes}
25.08.2016 Patch durch Infos mittels tracing aus Linux Kernel geschrieben, 0xDD als divisor, sonst gleich zu 8250-Treiber in jailhouse
.gitconfig mit [sendemail] eingerichtet
Rebase mit git durchgeführt, historie überarbeitet um E-Mail zu ändern (rebase -i -p HEAD~2; git commit --amend --author "Name <Mail>"; git rebase --continue)
Erste Gehversuche mit RTEMS - kein Port für jetson-tk1 oder HiKey vorhanden
