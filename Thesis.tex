%Dies ist nur ein Muster, es wird keine Gewährleistung auf Korrektheit gegeben! 

% Format des Dokuments
%    article - Kleinere Schriftwerke, ohne Titelblatt und nur mit einseitigem Druck
%    report - Mittlere Schriftwerke, mit Titelblatt und einseitigem Druck
%    book   - Große Schriftwerke, mit Titelblatt und zweiseitigem Druck
%    letter - Für Briefe 
\documentclass[numbers=noenddot, a4paper, 12pt]{scrreprt}

% Deutsche Umlaute
\usepackage[utf8]{inputenc}

% deutsche Silbentrennung
\usepackage[ngerman]{babel}

% Einbinden von Bildern
\usepackage{graphicx}

% Unterstützung von Hyperlinks
\usepackage{url}
\usepackage{hyperref}

% Mathematische Pakete
\usepackage{amsmath}
\usepackage{amssymb}
\usepackage{amsfonts}
\usepackage{amsthm}
\usepackage{latexsym}
\usepackage{morefloats}

\usepackage{geometry} 

%zeilenabstand
\usepackage{setspace}
\onehalfspacing


% better looking tables
\usepackage{ifthen}
\usepackage{booktabs}
\usepackage{multirow}

\newcommand{\forloop}[5][1]{%
\setcounter{#2}{#3}%
\ifthenelse{#4}{#5\addtocounter{#2}{#1}%
\forloop[#1]{#2}{\value{#2}}{#4}{#5}}%
{}}

\newcounter{crcounter}

\newcommand{\compensaterule}[1]{%
\forloop{crcounter}{1}{\value{crcounter} < #1}%
{\vspace*{-\aboverulesep}\vspace*{-\belowrulesep}}}

\newcommand{\multirowbt}[3]{\multirow{#1}{#2}%
{\compensaterule{#1}#3}}

% inline aufzählungs
\usepackage{paralist}

% farbdefinitionen
\usepackage{xcolor}
\usepackage{color}

% ausgabe von quelltext
\usepackage{textcomp}
\usepackage{listings}


\lstset{
%	backgroundcolor=\color{lbcolor},
	tabsize=4,    
%	rulecolor=,
	language=[GNU]C++,
	basicstyle=\scriptsize,
	upquote=true,
	%aboveskip={1.5\baselineskip},
	columns=fixed,
	showstringspaces=false,
	extendedchars=false,
	breaklines=true,
	%prebreak = \raisebox{0ex}[0ex][0ex]{\ensuremath{\hookleftarrow}},
	%frame=single,
	numbers=none,
	showtabs=false,
	showspaces=false,
	showstringspaces=false,
	identifierstyle=\ttfamily,
	language=C++,
    basicstyle=\ttfamily,
    keywordstyle=\color{blue}\ttfamily,
    stringstyle=\color{red}\ttfamily,
    commentstyle=\color[rgb]{0,.7,0}\ttfamily,
    morecomment=[l][\color{magenta}]{\#},
	%\lstdefinestyle{C++}{language=C++,style=numbers}’,
}
\lstset{
	basicstyle=\renewcommand{\baselinestretch}{.9}\ttfamily,
	xleftmargin=.5cm,
	language=C++,
	numbers=left,
	stringstyle=\color{black}\ttfamily,
	keywordstyle=\color{black}\ttfamily,
	numberstyle=\tiny}
% acronyme
\usepackage[printonlyused]{acronym}

% \enquote{quoted text} -> yay
\usepackage{csquotes}

% coolere referenzen
\usepackage[german]{fancyref}

% tabellen, formeln und abbildungen ohne chapter praefix
\usepackage{chngcntr}
\counterwithout{table}{chapter}
\counterwithout{equation}{chapter}
\counterwithout{figure}{chapter}


% todo commando, faerbt todos rot ein und schreibt sie uppercase
\newcommand{\todo}[1]{\textcolor{red}{(TODO: \MakeUppercase{#1})}}


\begin{document}

%Definiert das Titelblatt
\pagenumbering{Roman}
\begin{titlepage}
\titlehead{
	\centering
	\large \textbf{Ostbayerische Technische Hochschule Regensburg \\
    				Fakultät Informatik und Mathematik}\\
	\vspace{1.5cm}     
	\includegraphics[width=7cm]{./images/HS-Logo_alpha}
	\vspace{-1mm}
	\title{\LARGE TITLE}
	\subtitle{\large Eingereicht zur Erlangung des akademischen Grades \\ \large Bachelor of Science (B. Sc.)}
	\author{\large AUTHOR\\ \large Matr.Nr.: braucht man wahrscheinlich nicht\\ \large Studiengang: IN}
	\publishers{\vspace{-1mm}\large Betreut durch Prof. Dr. rer. nat. ...} %\\2. Gutachter: Prof. Dr. Dumbledore}
}
\date{\vspace{0mm}\large\today}
\end{titlepage}

%Setzt den Titel
\maketitle

% Inhaltsverzeichnis anzeigen
%\parskip am schluss anpassen, so dass ein zweiseitiges inhaltsverzeichnis gut umgebrochen wird
{\parskip=+5mm \tableofcontents}

%abkürzungsgroup
\chapter*{Abkürzungsverzeichnis}
\addcontentsline{toc}{chapter}{Abkürzungsverzeichnis}
\begin{acronym}
\acro{BSD}{Berkeley Software Distribution}
\end{acronym}
\newpage

\newcounter{romanPagenumber}
\setcounter{romanPagenumber}{\value{page}}
\pagenumbering{arabic}

\input{chapter1}
\chapter{kapitel 2}
\section{IDE einrichten}
http://stackoverflow.com/questions/33676829/vim-configuration-for-linux-kernel-development
http://vim.wikia.com/wiki/Vim\_buffer\_FAQ
\section{Function tracing}
Um herauszufinden, welche Funktionen beim 'Tegra TK1' für die UART-Baudrate zuständig ist, habe ich mir die relevanten Funktionen ausgeben lassen. Da als Plattform 'ARM' eingesetzt wird, befindet sich im Ordner 'arch/arm/boot/dts' im Linux Kernel Projekt die Namen, welche anzeigen, für welches Bauteil welcher Treiber bei welcher Adresse benutzt wird. In der Datei 'tegra124.dtsi' lässt sich dann herausfinden, dass der Uart-Treiber auf dem '8250'-Chip basiert. In 'drivers/tty/serial/8250' sind die relevanten Quelldateien vorhanden. Um zu sehen, wie der Linux-Kernel die Baudrate ändert, lässt sich als zu filternder Funktionsnamenspräfix \enquote{serial8250} ausmachen. 	

Dafür müssen im Linux Kernel unter \enquote{Kernel Hacking -> Tracers} die Optionen für \enquote{Kernel Function Tracer}, \enquote{Kernel Function Graph Tracer} und \enquote{enable/disable function tracing dynamically} auf \enquote{true} gesetzt werden. Anschließend ist es möglich, die Traces in einen eingehängten Ordner schreiben zu lassen. Hierfür kann man entweder das Dateisystem zur Laufzeit mittels 'mount -t debugfs nodev /sys/kernel/debug' oder zum Betriebssystemstart mittels \enquote{echo "debugfs       /sys/kernel/debug          debugfs defaults        0       0" >> /etc/fstab} einhängen. 

Danach schalte ich das Tracing mittels \enquote{echo 0 > /sys/kernel/debug/tracing/tracing\_on} aus, um irrelevante Einträge zu vermeiden. 

echo function\_graph > current\_tracer

echo serial8250* > set\_ftrace\_filter

echo 1 > tracing\_on

cat trace\_pipe

\section{Uart driver}
Speziellen divisor setzen:
Clk speed / baud << 4
Sense Value register is updated with the count value. The low 20
bits of the ASR give the number of clocks within a single bit. Because the UART uses 16x oversampling, the resulting value
needs to be adjusted by shifting right 4 bits, then loading the resulting count in the divisor latch of the UART. (In the code
snippet below, the lower 4 bits are rounded to give slightly greater accuracy.)
(Technical Reference Manual 2408)
408000000 / 16 * 115200 (1843200) = 221

\section{rtems}
Kein Support für Tegra
git clone git://git.rtems.org/rtems.git rtems 
cd rtems
\subsection{rtems-source-builder}
mkdir src
cd src
git clone git://git.rtems.org/rtems-source-builder.git
cd rtems-source-builder
source-builder/sb-check
../source-builder/sb-set-builder --log=l-arm.txt --prefix=\$(pwd)/4.12 4.12/rtems-arm

\section{Linaro-Hikey}
Zwei Anleitungen führen zum Ziel, eine zur Verzweiflung XD
Erst uefi, dann der Rest:
https://github.com/96boards/documentation/wiki/HiKeyUEFI
http://wiki.lemaker.org/HiKey(LeMaker\_version):Building\_Debian\_from\_Source\_Code
dir=\$(pwd); cd /usr/bin; sudo mv python python.bck; sudo ln -s python2.7 python; cd \$dir; \${UEFI\_TOOLS\_DIR}/uefi-build.sh -b RELEASE -a ../arm-trusted-firmware hikey -c ../LinaroPkg/platforms.config; cd /usr/bin/; sudo mv python.bck python; cd \$dir

\section{Notes}
25.08.2016 Patch durch Infos mittels tracing aus Linux Kernel geschrieben, 0xDD als divisor, sonst gleich zu 8250-Treiber in jailhouse
.gitconfig mit [sendemail] eingerichtet
Rebase mit git durchgeführt, historie überarbeitet um E-Mail zu ändern (rebase -i -p HEAD~2; git commit --amend --author "Name <Mail>"; git rebase --continue)
Erste Gehversuche mit RTEMS - kein Port für jetson-tk1 oder HiKey vorhanden

\input{chapter3}

\newpage

\pagenumbering{Roman}
\setcounter{page}{\theromanPagenumber}

\appendix

%bibgroup
\begingroup
\let\cleardoublepage\relax
\let\clearpage\relax
\newpage
\chapter{\bibname}
\renewcommand{\chapter}[2]{}%
\bibliography{literatur}%{\nocite{*}}
\bibliographystyle{unsrt}
\endgroup


%abbildungsgroup
\begingroup
\let\cleardoublepage\relax
\let\clearpage\relax
\newpage
\chapter{\listfigurename}
\renewcommand{\chapter}[2]{}%
\listoffigures
\endgroup

\restoregeometry
%Definiert und setzt die Eigenständigkeitserklärung
\chapter{Eigenständigkeitserklärung}
\vspace{15mm} 
Der Verfasser erklärt, dass er die vorliegende Arbeit selbständig, ohne fremde Hilfe und ohne Benutzung anderer als der angegebenen Hilfsmittel angefertigt hat. Die aus fremden Quellen (einschließlich elektronischer Quellen) direkt oder indirekt übernommenen Gedanken sind ausnahmslos als solche kenntlich gemacht. Die Arbeit ist in gleicher oder ähnlicher Form oder auszugsweise im Rahmen einer anderen Prüfung noch nicht vorgelegt worden.


%% Abstand und Linie
\vspace{20mm}
\hrule
\vspace{5mm}
Regensburg, \date{\today}
 
\vspace{5cm}



\end{document}
